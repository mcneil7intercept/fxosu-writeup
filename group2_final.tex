\documentclass[12pt]{article}
\usepackage[top=2in, bottom=1.5in, left=1in, right=1in]{geometry}
\usepackage{pdfpages}
\usepackage{fancyhdr}
\usepackage{pgfplots}
\usepackage{hyperref}
\usepackage[T1]{fontenc}
\usepackage{graphicx}
\graphicspath{ {images/} }
\pagestyle{fancy}
\fancyhf{}
\fancyhead[LE,RO]{NETWORK RELIABILITY API FOR FIREFOX OS}
\fancyfoot[CE,CO]{\leftmark}
\fancyfoot[LE,RO]{\thepage}

\begin{document}
\title{NETWORK RELIABILITY API FOR FIREFOX OS\\Improving Efficiency Through Analysis of Network Metadata}
\maketitle
\begin{center}
    \author{John Zeller\\Pok Yan Tjiam\\Jonathan McNeil}
\end{center}
\pagebreak

\tableofcontents
\pagebreak

\section{Introduction}
% Who requested it?
% Why was it requested?
% What is its importance?
% Who was/were your client(s)?
% Who are the members of your team?
% What were their roles?
% What was the role of the client(s)? (I.e., did they supervise only, or did they participate in doing development)

This project was requested by Mozilla, specifically the team working on Firefox OS. Our customer during the course of our project was Dietrich Ayala, a Project Manager working on Firefox OS and a member of the Mozilla family since February 2006. Dietrich's role in our project took many forms, and evolved as what we needed did so. Initially, he helped on board us to the Firefox OS platform, providing us with plenty of documentation to wrap our heads around, and once we have devoured this information, he was quick to setup meeting with members form the Firefox OS networking team. Throughout the year Dietrich was always available via IRC, Skype, Email, and Cell, whenever we needed him. And if he could not answer a question for us, he knew who could.
\\\\
The members of our team were John Zeller, Pok Yan Tjiam, and Jonathan McNeil. John Zeller, having worked with Mozilla for over a year as an intern and contractor, acted as a de facto team lead during the course of the project. Pok Yan Tjiam worked on \_\_\_. Jonathan McNeil worked on \_\_\_.
\\\\
The purpose of this project was to help work towards giving developers a tool that allows them to avoid the typical environment agnostic network request paradigm. As it currently stands, most developers create apps with little to no information about the quality of the network. In their apps, they typically run network requests in a dumb loop, which simply tries continually until the request succeeds. This is wasteful in terms of both battery and data, and overloads an often times already overloaded network.
\\\\
Firefox OS is not a direct competitor to the well known giants in the mobile OS space, but is aimed at being a good solution for the next 4 billion people coming online; the majority of which are coming online in developing countries where network infrastructure is literally decades behind, having a profoundly negative impact on user experience. In many cases poor network conditions can even render some applications non-functional. Aside from out-dated networks, some areas of these developing countries also have unreliable electrical grids, which can make charging a phone take days for many users living with the most unreliable grids. And this problem is only growing worse, at an ever growing pace. In the next 30 days, India alone is expected to account for 5 million new internet users.

\section{Requirements Document}
\subsection{Original}
Here is our original requirements document, featuring the details of our project as we understood them at the time it was written.
\includepdf[pages={1-8}]{images/requirementsdoc.pdf}

\subsection{Revisions}
% What new requirements were added? Why? 
% What existing requirements were changed? Why? 
% What existing requirements were deleted? Why? 
% Table featuring changed requirements
	% 1 | Requirement | What happened to it | Comments
% What was the final Gantt chart?

\subsubsection{Requirements}
As happens in most projects, requirements change, and ours was no different. Here are the changes that were made to our requirements and why.\\\\
\begin{tabular}{|l|l|l|l|}
\hline
\#  & Requirement 				& What Happened To It 		  & Comments \\ \hline
4	& The API should be written & Modified 					  & The reason for this... \\
	& in C++.					& Changed to "The API should  & \\
	&							& be written in C++ and/or	  & \\
	&							& JavaScript" 				  & \\ \hline
5 	& The API should be callable& Modified 					  & The reason for this... \\
	& by JavaScript executing in& First became "The API should& \\
	& the DOM in a web context. & be callable by JavaScript   & \\
	&							& executing in the window.",  & \\
	&							& Then modified again to 	  & \\
	&							& become "The prototype API   & \\
	&							& should be callable by 	  & \\
	&							& JavaScript executing in a   & \\
	&							& web sandbox." 			  & \\ \hline
6 	& The prototype API should  & Removed 				 	  & The reason for this... \\
	& be able to queue tasks. 	& New requirement was "The 	  & \\
	& 						 	& prototype API should be able& \\
	&							& be able to access data on   & \\
	&							& the battery level of the    & \\
	&							& device in order to determine& \\
	&							& if a task should be 		  & \\
	&							& executed." 				  & \\ \hline
7 	& The prototype API should  & Removed				 	  & The reason for this... \\
	& be able to prioritize		& New requirement was "The 	  & \\
	& tasks in the queue based	& protoype API should be able & \\
	& on network, battery and 	& to access data on the 	  & \\
	& system data.				& charging state of the device& \\
	&							& in order to determine if a  & \\
	&							& task should be executed."   & \\ \hline
\end{tabular}
\pagebreak

\begin{tabular}{|l|l|l|l|}
\hline
\#  & Requirement 				& What Happened To It 		  & Comments \\ \hline
8 	& The prototype API should	& Modified 					  & The reason for this... \\
	& be developer configurable,& Changed to "The prototype   & \\
	& to provide a force		& API should be developer 	  & \\
	& execution of a task after	& configurable, to provide a  & \\
	& a period of time.			& level of certainty about 	  & \\
	&							& network quality."			  & \\ \hline
9 	& The prototype API should  & Modified 				      & The reason for this... \\
	& be able to function  	   	& Changed to "The prototype   & \\
	& without error on Firefox  & API should be able to 	  & \\
	& OS, Firefox for Android,	& function without error on   & \\
	& and Firefox for Desktop.	& Firefox for Desktop."		  & \\ \hline
10 	& The API should be able to & Modified 				      & The reason for this... \\
	& access data on the quality& Changed to "The prototype   & \\ 
	& of the network in order to& API should be able to access& \\
	& determine if a task should& latency-related network     & \\
	& be executed.				& information to determine if & \\
	&							& a task should be executed." & \\ \hline
16 	& The API should be able to & Removed					  & The reason for this... \\
	& queue and postpone tasks	& New requirement was "The API& \\
	& from being dispatched.	& should be able to access 	  & \\
	&							& latency-related network 	  & \\
	&							& information to determine if & \\
	&							& a task should be executed." & \\ \hline
17 	& The API should be able to & Removed 					  & The reason for this... \\
	& prioritize queued tasks 	& New requirement was "The API& \\
	& based on accessible data  & should take into account the& \\
	& about the device and the  & type of network connection, & \\
	& environment.				& whether it be wifi, cellular& \\
	&							& data, etc."				  & \\ \hline
18	& The API should have a 	& Removed					  & The reason for this... \\
	& mechanism for dispatching & New requirement was "The API& \\
	& queued tasks.				& should be able to access 	  & \\
	&							& data about recent tx/rx 	  & \\
	&							& data."					  & \\ \hline
\end{tabular}
\pagebreak

\begin{tabular}{|l|l|l|l|}
\hline
\#  & Requirement 				& What Happened To It 		  & Comments \\ \hline
19 	& The API should be  	    & Modified 					  & The reason for this... \\
	& developer configurable, to& Changed to "The API should  & \\
	& prioritize application 	& be developer configurable,  & \\
	& specific tasks.			& to provide a level of 	  & \\
	&							& certainty about network 	  & \\
	&							& quality."					  & \\ \hline
20 	& The API should be  	    & Removed					  & The reason for this... \\
	& developer configurable,   & New requirement was "The 	  & \\
	& to provide a force  	    & prototype API should be able& \\
	& execution of a task after & to access data about recent & \\
	& a period of time.			& tx/rx data to determine if a& \\
	&							& task should be executed."	  & \\
	&							& Then, the requirement was   & \\
	&							& modified to become "The 	  & \\
	&							& prototype API should be able& \\
	&							& to see if the device has an & \\
	&							& internet connection to 	  & \\
	&							& determine if a task should  & \\
	&							& be executed."				  & \\ \hline
\end{tabular}
\pagebreak

\subsubsection{Gaant Chart}
Here is the final Gaant chart for how our project timeline actually turned out.

\begin{figure}[h!]
  \centering
	\includegraphics[scale=0.3]{finalgaant1.png}
  \caption{1 of 3: The final Gaant chart for how our project timeline actually turned out.}
\end{figure}
\begin{figure}[h!]
  \centering
	\includegraphics[scale=0.3]{finalgaant2.png}
  \caption{2 of 3: The final Gaant chart for how our project timeline actually turned out.}
\end{figure}
\begin{figure}[h!]
  \centering
	\includegraphics[scale=0.3]{finalgaant3.png}
  \caption{3 of 3: The final Gaant chart for how our project timeline actually turned out.}
\end{figure}
\pagebreak

\section{Weekly Team Blog Posts}
% These should be formatted nicely and clearly distinct from one another.
{\setlength{\parindent}{0cm}
\hrulefill \\
Friday, October 31, 2014 \\
Fall 2014, Week 5: \\

Plans for coming week: \\
- Start working on the requirements documemt \\
- Continues to research on Firefox OS \\

Progress since last week: \\
- Problem statement finished \\
- Technology review finished \\
- Project blog and project site set up \\
- Problem statement and technology review uploaded to the project site \\

This week we set up our project blog and project site using Sharepoint space. The project site is used to store and share documents we made such as the problem statement and technology review to the whole team. The project blog is used to keep track of our progress on this project. We posted our progress updates on the project blog every week so that our client can know what we are currently working on. \\

\hrulefill \\
Friday, November 7, 2014 \\
Fall 2014, Week 6: \\

Plans for coming week: \\
- Continues to work on the requirements document \\
- Continues research (necko), get familiar with Web IDL specifications \\

Progress since last week: \\
- Finished the basic structure of the requirements document \\

We expect to finish the introduction and project description parts of the requirements document, and think of some requirements for this project in the next week. We create our first draft of the requirement document in Word format instead of LaTeX because it is easier for us to edit it online. We stored our requirement document on Google Drive and shared it to the whole team as well as our client so that every one in our team can work on the document at the same time. Also, our client can give us immediate feedback without sending email but simply add a line somewhere on the document. We followed the format provided in the assignment page and finished the basic structure of the requirements document at the end of this week. \\

\hrulefill \\
Friday, November 14, 2014 \\
Fall 2014, Week 7: \\

Plans for coming week: \\
- Continues to work on the requirements document \\
- Finish the requirment parts of the requirements document \\

Progress since last week: \\
- Finished the introduction part of the requirements document \\

We continue to work on the requirement document in this week. We combined the words FxOS and OSU to FxOSU and decided to use it as our team name. Our progress on the requirement document was pretty slow because we have to do more research on Firefox OS at the same time. At the end of this week we finished the introduction part of the requirements document and filled in some requirements of the project. \\

\hrulefill \\
Friday, November 21, 2014 \\
Fall 2014, Week 8: \\

Plans for coming week: \\
- Finish the whole requirements document \\
- Get the requirements document signed \\
- start working at the poster \\

Progress since last week: \\
- Finished the project description part of the requirements document \\

This week we were still working on the requirement document. We were a bit behind schedule and need to finish the requirement document as soon as possible in order to get feedback from our client and get it signed before Thanksgiving. \\

\hrulefill \\
Friday, November 28, 2014 \\
Fall 2014, Week 9: \\

Plans for coming week: \\
- Finish the poster \\
- Start working on the preliminary design document \\

Progress since last week: \\
- Requirements document finished and submited \\
- Requirements document uploaded to the project site \\
- Start working on the poster \\

Finally we finished the requirements document and got it signed. The requirements document was also uploaded to the project site. We felt like we have a problem on our time management. To prevent similar thing happens, we need to check each other’s schedule and have more time to work together. \\

\hrulefill \\
Friday, December 5, 2014 \\
Fall 2014, Week 10: \\

Plans for coming week: \\
- Finish the preliminary design document \\
- Finish the progress report \\
- Meet with Professor McGrath for end-of-term meeting \\

Progress since last week: \\
- Preliminary Poster finsihed \\
- Preliminary Poster uploaded to the project site \\

This week we started working on the poster that we will be using in the Expo and got it finished in this week. Since this is only a preliminary design of the poster, we did not fill in all the contents. The poster have a blue background and orange header and it follows the Firefox OS branding guidelines. \\

Two of our member, John and Jonathan, went to Portland to meet with our client and engineers from Mozilla. We have a more clear direction on what we are going to do. \\

\hrulefill \\
Friday, December 12, 2014 \\
Fall 2014, Finals week: \\

Plans for the next term: \\
- Working on the Javascript prototype \\
- Research on Services Worker \\
- Research on RequestSync \\

Progress since last week: \\
- Preliminary design document finished \\
- Progress report finished \\

This term we spent most of the time working on documents and research. Next term we will start working on the prototype of our API. and we will later move on to the implementation in C++. \\

This week is the end of Fall term. The blog will stop updating until next term. \\

\hrulefill \\
Friday, January 9, 2015 \\
Winter 2015, Week 1: \\

Plans for coming week: \\
- Start working on the actual implementation \\

Progress since last week: \\
- Set up the weekly meeting with our TA \\
- Set up our group meeting time \\
- Requirements updated \\

Winter term begins and we get back to our project. In the first week, we scheduled a weekly meeting with our TA. and updated the requirements. We broke down our project into 24 requirements and each member handle 8 of them. We are expected to complete 6 requirements at the end of this term and 2 for the next term. \\

\hrulefill \\
Friday, January 16, 2015 \\
Winter 2015, Week 2: \\

Plans for coming week: \\
- Start working on the prototype API \\

Progress since last week: \\
- Updated requirements approved \\
- Requirements completion list submitted \\

This week we got our new requirements approved and divided works to each member. We also started our implementation of prototype API using JavaScript. \\

\hrulefill \\
Friday, January 23, 2015 \\
Winter 2015, Week 3: \\

Plans for coming week: \\
- Working on the prototype API \\

Progress since last week: \\
- Updated requirements approved \\

Again we update one of the requirement to make it more specific and got it approved. After speaking with our sponsor, we decided to implement our prototype API as a Firefox addon instead of putting it into the Firefox core, while the  C++ API will still need to be put into the core. \\

\hrulefill \\
Friday, January 30, 2015 \\
Winter 2015, Week 4: \\

Plans for coming week: \\
- Finish the implementation of the prototype API \\
- Finish the first 9 requirements and get ready for demo \\

Progress since last week: \\
- Created a draft of the prototype API \\

This week we were working on the prototype API. Because our API will communicate between C++ and Javascript, we implement our prototype API as a XPCOM component. We succesfully installed the prototype API as an addon to the Firefox but we failed to call the C++ function in Javascript. We will keep working on it and try to make the addon works in the next week. \\

\hrulefill \\
Friday, February 6, 2015 \\
Winter 2015, Week 5: \\

Plans for coming week: \\
- Finish the implementation of the prototype API \\
- Test the prototype API on desktop, Android and Firefox OS \\
- Demo the prototype API to the TA \\

Progress since last week: \\
- Implemented the prototype API addon using addon SDK \\
- Battery level, device charging state and network latency data are accessible from the prototype API \\

This week we were still working on the prototype API. Since the addon-via-manifest prototype API does not works well, we tried another method and implemented the addon via SDK. Also, it turns out that some of the functionality of our API are only available on the Firefox OS but not the desktop version. Therefore we need to write a new prototype for the Firefox OS device. \\

\hrulefill \\
Friday, February 13, 2015 \\
Winter 2015, Week 6: \\

Plans for coming week: \\
- Start working on the actual API using C++ or Javascript \\

Progress since last week: \\
- Prototype API finished \\
- Completed the demo of first 9 requirements to the TA \\

This week we finally finished the prototype API. The prototype can detect the battery level, charging state of the device, network connection type and shows latency-related information. We also changed our requirements after talking to the sponsor to scope the prototype down so that we have more time to focus on the actual API. \\

\hrulefill \\
Friday, February 20, 2015 \\
Winter 2015, Week 7: \\

Plans for coming week: \\
- Continue working on the actual API using C++ or Javascript \\

Progress since last week: \\
- Start working on the actual API using C++ or Javascript \\

This week we keep on working on the API. A few weeks before we got a developer reference phone called Flame from our sponsor. We began testing our API on this Firefox OS device to make sure our API works on all platform of Firefox. \\

\hrulefill \\
Friday, February 27, 2015 \\
Winter 2015, Week 8: \\

Plans for coming week: \\
- Continue working on the API using C++ or Javascript \\

Progress since last week: \\
- Continue working on the API using C++ or Javascript \\

This week we are still working on the API. Most of the functions of the actual API are same as the prototype API. Some of the new functions of the actual API include collect network status information, access system load information and to integrate with existing API such as ServiceWorker and RequestSync. Each of our group member take charge of one of these function. \\

\hrulefill \\
Friday, March 6, 2015 \\
Winter 2015, Week 9: \\

Plans for coming week: \\
- Finish the next 9 requirements \\
- Finish the implementation of the API \\

Progress since last week: \\
- Continue working on the implementation of the API \\

We are going to demo the next 9 requirements to the TA in the final week. We hope to finish the implementation of our API next week and test it before we demo. \\

\hrulefill \\
Friday, March 13, 2015 \\
Winter 2015, Week 10: \\

Plans for coming week: \\
- Finish the next 9 requirements \\
- Finish the implementation of the API \\
- Demo the 9 requirements to the TA \\

Progress since last week: \\
- Poster updated \\
- Continue working on the implementation of the API \\

This week we updated the poster and keep working on the API. We are going to demo the requirements next week but some of the requirement are still incomplete. Some of the functions of our API took longer than we expected to implemented. \\

\hrulefill \\
Friday, March 20, 2015 \\
Winter 2015, Finals week: \\

Plans of the next term: \\
- Finish the implementation of the API \\
- Prepare for Expo \\

Progress since last week: \\
- Finished the demo of winter term requirements \\

This term we finished most of the implementation of the API.  We tested our API mainly on Firefox desktop. In the next term we will focus on the implementation and testing of our API on Firefox OS. \\

\hrulefill \\
Friday, April 3, 2015 \\
Spring 2015, Week 1: \\

Plans for coming week: \\
- Continue to work on the FxOSUService API \\

Progress since last week: \\
- Set up the weekly meeting with our TA \\

Spring term begins. This week we scheduled a weekly meeting with our TA. We have to complete the last 6 requirements before the Expo and prepare for the Expo onMay 15th. \\

\hrulefill \\
Friday, April 10, 2015 \\
Spring 2015, Week 2: \\

Plans for coming week: \\
- Continue to work on the FxOSUService API \\

Progress since last week: \\
- Met with our sponsor via video call \\

This week we had a meeting with our sponsor. We keep on working on our API and hope to finish it quickly so that we have time to work on the test app. \\

\hrulefill \\
Friday, April 17, 2015 \\
Spring 2015, Week 3: \\

Plans for coming week: \\
- Continue working on the FxOSUService API \\
- Working on the test app \\

Progress since last week: \\
- Updated the poster \\

This week we are still working on the requirements. We also start working on a test app that benchmark the performance of our API. The test app is implemented in Javascript. We also updated our poster. \\

\hrulefill \\
Friday, April 24, 2015 \\
Spring 2015, Week 4: \\

Plans for coming week: \\
- Continue working on the FxOSUService API \\
- Continue Working on the test app \\
- Update the poster \\

Progress since last week: \\
- Modify the format of our poster \\

This week we need to modify the poster to a specified format. It won't be hard because we only need to change the layout and content can stay the same. We need to finish our poster before May 1st. We are also working on the final draft of our API and hope to get it done by next week. \\

\hrulefill \\
Friday, May 1, 2015 \\
Spring 2015, Week 5: \\

Plans for coming week: \\
- Finish the test app \\
- Demo the requirements to our TA \\

Progress since last week: \\
- Finished the poster \\
- Finished our API \\

This week we finished the poster. The final draft of our FxOSUService API is finished and is working. Expo is approaching. We are going to demo our requirements to the TA and prepare for the Expo after the demo. \\

\hrulefill \\
Friday, May 8, 2015 \\
Spring 2015, Week 6: \\

Plans for coming week: \\
- Prepare for the Expo \\

Progress since last week: \\
- Finished all the requirements \\
- Finished the test app \\
- Finished the demo \\

This week we finished the test app and finished the demo of our requirements. Expo is next Friday and we have to start prepare for it. \\

\hrulefill \\
Friday, May 15, 2015 \\
Spring 2015, Week 7: \\

Plans for coming week: \\
- Start working on the report and presentation \\

Progress since last week: \\
- Expo is over \\

Expo is over. \\

\hrulefill \\
}

\section{Project Poster}
This is the final project poster that we used when demoing our project at the Engineering Expo.

\includepdf[pages={1},angle=90]{images/poster.pdf}


\section{Project Documentation}
\begin{enumerate}
    \item How does your project work?
    	\subitem What is its structure?\\
    	\subitem What is its Theory of Operation?\\
    	\subitem Block and flow diagrams are good here. 
    \item How does one install your software, if any?
    \item How does one run it?
    \item Are there any special hardware, OS, or runtime requirements to run your software?
    \item Any user guides, API documentation, etc. 
\end{enumerate}

\section{Learning New Technology}
% What web sites were helpful? (Listed in order of helpfulness.)
% What, if any, reference books really helped?
% Were there any people on campus that were really helpful? 
There were a number of websites that were helpful during the course of researching, designing, implementing and testing our project. Here is a list of the websites that we found most useful, organized from most to least helpful:
\begin{enumerate}
	\item Mozilla Developer Network (\href{https://developer.mozilla.org/en-US/}{https://developer.mozilla.org/en-US/})
	\item Mozilla Wiki (\href{https://wiki.mozilla.org/}{https://wiki.mozilla.org/})
	\item Boot2Gecko Mailing List (\href{https://groups.google.com/forum/\#!forum/mozilla.dev.b2g}{https://groups.google.com/forum/\#!forum/mozilla.dev.b2g})
	\item Gaia Mailing List (\href{https://groups.google.com/forum/\#!forum/mozilla.dev.gaia}{https://groups.google.com/forum/\#!forum/mozilla.dev.gaia})
	\item StackOverflow (\href{http://stackoverflow.com/}{http://stackoverflow.com/})
\end{enumerate}

\noindent Other resources that we found really helpful included:
\begin{enumerate}
	\item IRC (irc.mozilla.org)
		\subitem Find more info here: \href{https://wiki.mozilla.org/IRC}{https://wiki.mozilla.org/IRC}
\end{enumerate}

\section{What did we learn from all of this?}
\subsection{John Zeller}
\begin{enumerate}
    \item What technical information did you learn?
    \item What non-technical information did you learn?
    \item What have you learned about project work?
    \item What have you learned about project management?
    \item What have you learned about working in teams?
    \item If you could do it all over, what would you do differently? 
\end{enumerate}

\subsection{Pok Yan Tjiam}
\begin{enumerate}
    \item What technical information did you learn?
    \item What non-technical information did you learn?
    \item What have you learned about project work?
    \item What have you learned about project management?
    \item What have you learned about working in teams?
    \item If you could do it all over, what would you do differently? 
\end{enumerate}

\subsection{Jonathan McNeil}
\begin{enumerate}
    \item What technical information did you learn?
    \item What non-technical information did you learn?
    \item What have you learned about project work?
    \item What have you learned about project management?
    \item What have you learned about working in teams?
    \item If you could do it all over, what would you do differently? 
\end{enumerate}

\section{Appendix I}
\begin{enumerate}
	\item Essential Code Listings. You don't have to include absolutely everything, but if someone wants to understand your project, there should be enough here to learn from. If you worked within a larger project, something like a patch file might be a good way to go. 
\end{enumerate}

\section{Appendix II}
\begin{enumerate}
	\item Anything else you want to include. Photos, etc. 
\end{enumerate}

\end{document}